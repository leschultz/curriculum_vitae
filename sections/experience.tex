\section{Technical Experience}

\noindent \textbf{Research Assistant \hfill Jun 2018 --- Aug 2024}\\
Computational Materials Group, UW-Madison \dotfill Madison, WI

\begin{itemize}
    \item \textbf{Machine Learning Domain of Applicability for Materials:} Developed a method to determine the domain of applicability for machine learning models in materials property prediction. Utilized kernel density estimation to measure dissimilarity between test and training data in feature space. This measure effectively distinguishes chemically distinct groups and correlates high dissimilarity with poor model performance and unreliable uncertainty estimates. Provided automated tools for researchers via a package on GitHub and PyPI.

    \item \textbf{Scientific Cluster Construction and Administration:} Assisted in building and managing two high-performance computing clusters using OpenHPC with Warewulf provisioning and OpenPBS queue management. Compiled materials research software including VASP, LAMMPS, and Python. Implemented Environment Modules to streamline software management, enhancing research efficiency and reproducibility.

    \item \textbf{Quantifying Metallic Glass Forming Ability:} Developed a high-throughput workflow for efficiently training machine learning interatomic potentials and simulating complex material properties. Developed predictive models for metallic glass forming ability using computed elemental properties and simulated features. Demonstrated that key trends in properties with glass forming ability aligned with prior research insights.

\end{itemize}

\noindent \textbf{Summer Undergraduate Research Experience \hfill May 2017 --- Aug 2017}\\
Computational Nuclear Engineering Research Group, UW-Madison \dotfill Madison, WI

\begin{itemize}
    \item Developed a Python toolkit for automating and visualizing direct accelerated geometry monte carlo geometries in VisIt.
    \item The Python PEP 8 coding style was adopted to facilitate easier reading and collaboration among users.
\end{itemize}

\noindent \textbf{Capstone Design Project \hfill Sep 2016 --- Apr 2017}\\
Undergraduate Research, Fort Lewis College \dotfill Durango, CO

\begin{itemize}
    \item Designed and built an exotic propulsion test stand with a team of 5 engineering students.
    \item Implemented electrostatic displacement mechanism and modeled system response with MATLAB.
\end{itemize}

\noindent \textbf{Summer Undergraduate Research Fellowship \hfill May 2016 --- Aug 2016}\\
Advanced Diagnostics and Propulsion Research Laboratory, Purdue \dotfill Durango, CO

\begin{itemize}
    \item Operated pressure vessels, X-ray tube sources, and high-speed cameras for analysis of two dimensional sprays.
    \item Constructed a protective lead enclosure for X-ray tube sources to shield operating personnel from excessive radiation exposure.
\end{itemize}

\noindent \textbf{Design Project \hfill Dec 2015 --- Apr 2016}\\
Undergraduate Research, Fort Lewis College \dotfill Durango, CO

\begin{itemize}
    \item Designed and developed an interchangeable sensor package for measurement of water temperature, oxygen reduction potential, pH, time, and global positioning system data.
    \item Designed to be durable, waterproof, and easy to use by attaching to rafts or kayaks, the sensor package enables citizen scientists to collect data along the rivers.
\end{itemize}
